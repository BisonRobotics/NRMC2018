\documentclass[]{book}


\usepackage[utf8]{inputenc}
\usepackage{amsmath}
\usepackage{amsfonts}
\usepackage{amssymb}
\usepackage{graphicx}
\usepackage[left=1.00in, right=1.00in, top=1.00in, bottom=1.00in]{geometry}

\usepackage{parskip}
\usepackage{enumitem}
\usepackage[table]{xcolor}
\usepackage{hyperref}

\usepackage[T1]{fontenc}
\usepackage[scaled=1.0]{DejaVuSansMono}
\usepackage{listings}


\hypersetup{
	colorlinks=true,
	hidelinks=true,
	breaklinks=true,
	linkcolor=blue,
	filecolor=magenta,      
	colorlinks=false,
	pdftitle={Localization Cape Manual},
	bookmarks=true,
	pdfpagemode=FullScreen,
}
\def\UrlBreaks{\do\/\do-}

\definecolor{codegreen}{rgb}{0,0.6,0}
\definecolor{codegray}{rgb}{0.5,0.5,0.5}
\definecolor{codepurple}{rgb}{0.58,0,0.82}
\definecolor{backcolour}{rgb}{0.965,0.973,0.980}
\lstdefinestyle{Default-Custom-Style}{
	backgroundcolor=\color{backcolour}, 
	xleftmargin=0.5cm,
	frame=tlbr,
	framesep=8pt,
	framerule=0pt,  
	%commentstyle=\color{codegreen},
	%keywordstyle=\color{magenta},
	%numberstyle=\tiny\color{codegray},
	%stringstyle=\color{codepurple},
	basicstyle=\footnotesize\ttfamily,
	breakatwhitespace=false,         
	breaklines=true,                 
	captionpos=b,                    
	keepspaces=true,                 
	numbers=none,                    
	numbersep=5pt,                  
	showspaces=false,                
	showstringspaces=false,
	showtabs=false,                  
	tabsize=2
}
\lstset{style=Default-Custom-Style}

% Code listing structure
%\begin{lstlisting}[language=bash]
%
%\end{lstlisting}


\title{Localization Cape Manual}
\author{Jacob Huesman}

\begin{document}
\let\cleardoublepage\clearpage
\maketitle
\tableofcontents


\chapter{Notes}
\section{Using this document}
All urls and table of contents entries are active. If you click on them they'll take you to the url or document location respectively.

The code view is copy and paste able only on particular pdf viewers. Firefox works.

If you contribute to this document, make sure you add your name to the authors list.

\section{Compiling this document}
This document was developed with TeXstudio and and compiled with the TeX Live distribution of LaTeX. The required packages on Ubuntu can be installed with
\begin{lstlisting}[language=bash]
sudo apt-get install texstudio texlive texlive-latex-recommended texlive-latex-extra texlive-fonts-extra
\end{lstlisting}
To find what Ubuntu package your Latex package lives in, do an
\begin{lstlisting}
apt-cache search <name of package>
\end{lstlisting}


\chapter{Beagle Configuration}
\section{Notes}
Sometimes you need to reload the udev rules before sshing
\begin{lstlisting}[language=bash]
sudo udevadm control --reload-rules
\end{lstlisting}


\section{Hardware}
\begin{itemize}%[noitemsep]
	\item BeagleBone Black
	\item Class 10 SD Card
\end{itemize}


\section{Configuration}
\subsection{Getting an OS}
Retrieve latest IoT (non-GUI) image from: \url{https://beagleboard.org/latest-images}.

Follow instructions available at \url{http://beagleboard.org/getting-started} for "Update Board with latest software. You can ignore the start your beagle section.

Configure udev rules for ssh over usb: \url{http://beagleboard.org/static/Drivers/Linux/FTDI/mkudevrule.sh}. Download and run.

After plugging in the usb cable you should now be able to ssh into the BeagleBone Black

\begin{lstlisting}[language=bash]
ssh debian@192.168.7.2 # password: temppwd
\end{lstlisting}

\pagebreak
\subsection{Configuring the OS}
\begin{lstlisting}[language=bash]
# ssh into the board
ssh debian@192.168.7.2 # password: temppwd

# change default passwords
sudo passwd debian # use brnrmc
sudo passwd root   # use brnrmc

# add nrmc account
sudo useradd -m nrmc
sudo passwd nrmc
sudo chsh -s /bin/bash nrmc
\end{lstlisting}

\subsection{Running Updates}
On the Beagle
\begin{lstlisting}[language=bash]
su -
route add default gw 192.168.7.1
\end{lstlisting}
On the host
\begin{lstlisting}[language=bash]
su -

# Replace wlan0 with the name of your internet facing interface
iptables --table nat --append POSTROUTING --out-interface wlan0 -j MASQUERADE

# Replace eth5 with the name of your beaglebone interface, if you run an ifconfig that
# interface should provide your computer with an ip address of 192.168.7.1
iptables --append FORWARD --in-interface eth5 -j ACCEPT

echo 1 > /proc/sys/net/ipv4/ip_forward
\end{lstlisting}



\section{Stuff to look into}
Building a custom image with just the stuff we need: \url{https://github.com/fhunleth/bbb-buildroot-fwup}, \url{https://elinux.org/BeagleBone_Operating_Systems}

\chapter{Localization Cape Requirements}
\section{Power Requirements}
\subsection{Pocket Beagle}
Should be the same as the BeagleBone Black. Minimum recommended is 5V @ 1.2A, recommended is 5V @ 2A, which includes room for USB peripherals. See: \url{http://beagleboard.org/support/faq}

The actual usage by the BeagleBone Black measured during a test done by Adafruit saw current usage of less than 500mA. Which makes sense since it can operate as a USB device. See: \url{https://learn.adafruit.com/embedded-linux-board-comparison/power-usage}. It might actually be lower since there everything is integrated on the pocket beagle.

\subsection{ELP VGA USB Camera Module}
According to the specs available on Amazon this camera doesn't exceed 160mA of power usage. See: \url{https://www.amazon.com/gp/product/B01DRG250Q/ref=oh_aui_search_detailpage?ie=UTF8&psc=1}. Since they are USB Devices we can expect no more than 5V @ 500mA. 

\subsection{Dynamixel XL-320}
See: \url{http://www.robotis.us/dynamixel-xl-320/} and
6~8.4V


\end{document}
